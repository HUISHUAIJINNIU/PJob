\PQUEUE{} is a software that accompanies \PHO{}.
It's main purpose is to help distributing \PHO{} calculation jobs within a pool of working machines.
Therefore it uses \Condor{} as underlying grid computation framework.\bb

To represent a \PHO{} calculation job, a novel file format called \nameref{section:pjob} is introduced.
\PJOB{} files can be created and edited using the \nameref{section:pjob-editor} which is also distributed together with \PHO{} and \PQUEUE{}.\bb

\PQUEUE{} enables the user to open \PJOB{} files and insert them into a queue - hence it's name.
Jobs are processed according to the sequence in this queue,
and either are executed locally or on a remote machine via \Condor.
In both cases, a job is calculated by running the command line version of \PHO{} on it.\bb

After successfully running the simulation contained in the \PJOB{} file,
\PQUEUE{} reads the \PJOB's results into the applications memory.
Then, after receiving results of multiple job instances, the user may write all results into one single CSV file for example
or visualize this potentially multidimensional data in a three dimensional view within \PQUEUE.\bb

Similar to \PHO{} and PScript, utilizing a ECMAScript engine \PQUEUE{} is also scriptable.
This enables the user to build and update the job queue automatically and with user defined behavior.
Doing long-winded optimizations by calculating further simulation parameters as a function of former results
is the main use case for \PQUEUE{} scripts.\bb



\subsection{Outline}

Because of \PJOB{} and its concepts being a prerequisite for \PQUEUE{},
this manual starts with a detailed description of the \PJOB{} file format in section \ref{section:pjob}.

\PJOB{} files have to be created and edited in order to have jobs \PQUEUE{} can work with.
Section \ref{section:pjob-cmd} covers \nameref{section:pjob-cmd}, a command line tool that does this job.
\nameref{section:pjob-editor} is its GUI equivalent and is described in section \ref{section:pjob-editor}.
For a quick introduction, you could skip the first chapters and start with section \ref{section:pjob-editor}
and learn by example.

Section \ref{section:pqueue} covers \PQUEUE's main use cases. 