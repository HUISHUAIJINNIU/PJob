\PJOBCMD\ is a command line tool for creating, manipulating and extracting \PJOB\ archive files.
It implements the format specification described in section \ref{pjob:archive_format} (\nameref{pjob:archive_format}).
It is possible to create a valid \PJOB{} file by creating all the necessary files in one directory
and adding that directory to a new \PJOB{} file with this tool.

\subsection{Example usage}
Suppose you have a directory \textit{My\_PJob} with contents according to the \PJOB{} format specification
(see section \ref{pjob:structure} \nameref{pjob:structure}).
\lstset{basicstyle=\color{white},backgroundcolor=\color{black}, language=bash}
\begin{lstlisting}
lucksus:~ nico$ ls
My_PJob

lucksus:~ nico$ ls My_PJob/
Resources
Version
parameterdefintions.xml
resultdefintions.xml

lucksus:~ nico$ ls My_PJob/Resources/
main.pscript
my_system.pho
\end{lstlisting}
You can create a \PJOB{} archive by using the \textit{--create} command line switch:
\begin{lstlisting}
lucksus:~ nico$ PJobCmdLine --verbose --create My_PJob/ --file my.pjob

Opening my.pjob was succesful.

Creating file my.pjob containing:
parameterdefintions.xml
resultdefintions.xml
Version
Resources/main.pscript
Resources/my_system.pho

lucksus:~ nico$ 
\end{lstlisting}\bb

If you want to extract the results after the \PJOB{} was executed
you could either extract all files with \textit{-x} or \textit{--extract},
or you could use \textit{--peek} (\textit{-p}) to extract the result files only.
\begin{lstlisting}
lucksus:~ nico$ PJobCmdLine -v -p Runs/run_20110322_1748 -f my.pjob -o my_results/

Opening my.pjob was succesful.

Extracting files:
/Users/nico/my_results/EyeOpeningPenalty.txt
/Users/nico/my_results/my_system.log
/Users/nico/my_results/parametercombination.xml

lucksus:~ nico$
\end{lstlisting}



\subsection{Command line parameters}
\PJOBCMD\ is controlled via multiple command line switches.
Every switch has a long version that has to be prefixed by two dashes (i.e. \textit{--help})
and an optional short, one character version that has to be prefixed by one dash (i.e. \textit{-h}).
The following list shows all command line parameters with their long version and their optional short version
separated by comma.
Parameters that require arguments, like \textit{--create} or \textit{--file} for example,
need these arguments to be the next following string.
The same list is printed by \PJOBCMD\ when called with the \textit{help} switch.
\begin{itemize}
	\item \textbf{about,?} shows the about page
	\item \textbf{add,a} adds a folder or directory to the specified file and flush() is called
	\item \textbf{create,c} creates a new .pjob-file which ONLY contains the specified folder. Only takes effect if the file is empty,
							doesn't exist or --force is used
	\item \textbf{detail,d} shows a detailed description of the specified command with examples if provided
	\item \textbf{extract,x} extracts all files
	\item \textbf{file,f} specifies the path of the .pjob-file. If it doesn't exist, an empty file will be created automatically					
	\item \textbf{force} sets overwrite-flag for all commands
	\item \textbf{help,h} produces help message
	\item \textbf{out,o} sets the target directory for --extract and --peek
	\item \textbf{peek,p} extracts the specified source files to --out path
	\item \textbf{verbose,v} displays detailed information about the tasks being made, e.g. list of extracted files
\end{itemize}
