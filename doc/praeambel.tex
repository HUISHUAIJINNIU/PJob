\documentclass[11pt,a4paper, onecolumn, twoside]{scrartcl}

\usepackage[latin1]{inputenc}
\usepackage{fancyhdr,amsmath,amsfonts, calc, tabularx, amssymb}
\usepackage{graphicx, float, paralist}	% Bilder einbinden können, die auch gleiten können
\usepackage{makeidx}	% Index erzeugen können
\usepackage[usenames,dvipsnames]{xcolor}	% Farben auch direkt ansteuern können per Name
\usepackage[font={ small}, labelsep=period,format=plain]{caption} % Bildunterschriften

\usepackage[linkbordercolor={1 1 1}, citebordercolor={1 1 1}, pdfborder={1 1 1}, plainpages=false]{hyperref}	% Hyperlinks korrekt einfärben
\usepackage[nottoc]{tocbibind}	% Inhaltsverzeichnistiefe vorgeben können
\usepackage{subfig}	% Subbilder erzeugen können
\usepackage{trsym}	% 
\usepackage{multirow}	% In Tabellen auch Inhalte über mehrere Zeilen definieren können
\usepackage{wrapfig}	% Abbildungen von Text umfließen lassen können
\usepackage{tabularx} % Tolle Tabellen, die automatisch die korrekte Breite haben
\usepackage{textcomp}	% "Mü" auch im normalen Text verwenden können!
\usepackage[]{bookmark}
\usepackage{wasysym, txfonts}	% Mathematische Elementsymbole (Complex, etc.)
%\usepackage{cite} % Zitate auch direkt angeben können, für z.B. [1-4]



% Mögliche Grafikformate, die man einbinden kann. Der Compiler sucht übrigens genau in dieser Reihenfolge,
% falls KEINE Endung im jeweiligen \includegraphics{} verwendet wird!
\DeclareGraphicsExtensions{.png, .jpg, .bmp, .pdf}

% Style-Datei liegt um übergeordneten Verzeichnis. In dieser Datei werden alle Format-relevanten
% Einstellungen vorgenommen. Diese betreffen das komplette Manual und sollten nur mit Bedacht geändert werden!
\usepackage{../TexManual}

%Matlab als Kommando anmelden, da es besondere Schreibweisen hat, genauso für PHOTOSS, Condor, etc.
\newcommand{\Matt}{\mbox{MATLAB\textsuperscript{\textregistered}}}
\newcommand{\MattZeile}{\mbox{MATLAB\textregistered}}
\newcommand{\MAT}{\Matt}

\newcommand{\PHO}{\mbox{PHOTOSS}}
\newcommand{\PS}{PScript}
\newcommand{\Condor}{Condor\textsuperscript{\textregistered}}
\newcommand{\CondorZeile}{Condor\textregistered}

% Name der ausführbaren PHOTOSS-EXE-Datei:
\newcommand{\PHOEXE}{\mbox{\lstinline!PHOTOSS.exe! }}
% Bigbreak als verkürztes Kommando:
\newcommand{\bb}{\bigbreak}
% Breite für die PHOTOSS-Parameter-Tabllen in pt als Platzhalter:
\newcommand{\PHOTableWidth}{440pt}

\newcommand{\PQUEUE}{\mbox{PQueue}}
\newcommand{\PJOB}{\mbox{PJob}}
\newcommand{\PJOBCMD}{\mbox{\PJOB Cmd}}
